\HeaderA{cowAnalyze}{Analyze time coded behavior data}{cowAnalyze}
\begin{Description}\relax
This function provides descriptive statistics from time coded behavior
datafiles recorded with CowLog.
\end{Description}
\begin{Usage}
\begin{verbatim}
cowAnalyze(file = NULL, states = NULL, events = NULL,
 state.names = NULL, event.names = NULL)
\end{verbatim}
\end{Usage}
\begin{Arguments}
\begin{ldescription}
\item[\code{file}] CowLog data file, or a file in same format
\item[\code{states}] A vector with the codes in the file that belong to states
\item[\code{events}] A vector with codes in the file that belong to events
\item[\code{state.names}] A character vector with the names for the states 
\item[\code{event.names}] A character vector with the names for the events
\end{ldescription}
\end{Arguments}
\begin{Value}
\begin{ldescription}
\item[\code{state}] Results for states
\item[\code{event}] Results for events
\end{ldescription}
\end{Value}
\begin{Author}\relax
Matti Pastell <matti.pastell@helsinki.fi>
\end{Author}
\begin{References}\relax
Hanninen, L. \& Pastell, M. CowLog: Open source software for coding
behaviors from digital video. Behavior Research Methods. 41(2),
472-476.

http://www.mm.helsinki.fi/~mpastell/CowLog
\end{References}
\begin{Examples}
\begin{ExampleCode}
##Analyze CowLog datafile named calf1.bh1,
## codes 1-3 are states and codes 4-5 are states.
## The names for the states are lying, standing, walking.
## Not run: 
analyzed <-cowAnalyze(file='calf1.bh1',states=c(1,2,3),
events=c(4,5),state.names=c('lying','standing','walking'))
## End(Not run)
 
\end{ExampleCode}
\end{Examples}

