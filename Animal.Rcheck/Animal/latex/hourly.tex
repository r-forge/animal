\HeaderA{hourly}{Calculate hourly values from time series}{hourly}
\begin{Description}\relax
Calculate hourly values (e.g mean or sum) from time series data. Allows
to specify a subject to calculate hourly values for several subjects.
\end{Description}
\begin{Usage}
\begin{verbatim}
hourly(data, time, fun = sum, subject = NULL)
\end{verbatim}
\end{Usage}
\begin{Arguments}
\begin{ldescription}
\item[\code{data}] A data vector that you want to calculate the hourly values
for
\item[\code{time}] Time stamps for data in POSIXct format
\item[\code{fun}] The function to apply, defaults to sum
\item[\code{subject}] You can optionally specify to a subject. e.g. to get
hourly values for each cow in a herd.
\end{ldescription}
\end{Arguments}
\begin{Value}
A data.frame with following elements
\begin{ldescription}
\item[\code{Hour}] Hour 1-24
\item[\code{Subject}] Appears only if you have specified a subject
\item[\code{Result}] The result of the function
\end{ldescription}
\end{Value}
\begin{Author}\relax
Matti Pastell <matti.pastell@helsinki.fi>
\end{Author}
\begin{SeeAlso}\relax
\code{\LinkA{daily}{daily}}, \code{\LinkA{weekly}{weekly}},
\code{\LinkA{monthly}{monthly}}
\end{SeeAlso}
\begin{Examples}
\begin{ExampleCode}
data(RIC)
RIC2 <- clean.RIC(RIC)
#Hourly feed intake of a whole from data set RIC
herd <- hourly(RIC2$intake,time=RIC2$begin,fun=sum)
#Hourly feed intake of individual cows from data set RIC
herd <- hourly(RIC2$intake,time=RIC2$begin,fun=sum,subject=RIC2$cowID)
\end{ExampleCode}
\end{Examples}

