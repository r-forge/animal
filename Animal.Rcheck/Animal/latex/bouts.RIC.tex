\HeaderA{bouts.RIC}{Merge bouts from RIC roughage intake files}{bouts.RIC}
\begin{Description}\relax
Merges single visits from roughage intake log file if the time
difference between successive visits is less than specified time
difference i.e merges multiple rows in the log file that are considered
to be a single feeding bout to a single row.
\end{Description}
\begin{Usage}
\begin{verbatim}
bouts.RIC(data, bout.diff = 5)
\end{verbatim}
\end{Usage}
\begin{Arguments}
\begin{ldescription}
\item[\code{data}] A data.frame read in with read.RIC, with clean=T option
\item[\code{bout.diff}] The maximum time difference (in minutes) between visits in a
single bout
\end{ldescription}
\end{Arguments}
\begin{Value}
A data.frame with the values merged for individual bouts. All other
objects are self.explanatory, but there are two that need further
clarification  
\begin{ldescription}
\item[\code{bout.duration}] This is the duration in minutes of the merged bout
i.e. begin-end for the bout.
\item[\code{intake.duration}] This is the time in minutes that the cow has kept her head
in the feeding trough during the bout. You may want to use this for
calculating feeding speed (kg/min)
\end{ldescription}
\end{Value}
\begin{Note}\relax
The function is currently only implemented for merging feeding
time and intake, thus protein, ash etc. are dropped from the resulting
datafrane. The variables in the code are finnish and thus maybe difficult to
follow. An english translation will (possibly) appear in the future.
\end{Note}
\begin{Author}\relax
Matti Pastell <matti.pastell@helsinki.fi>
\end{Author}
\begin{SeeAlso}\relax
\code{\LinkA{read.RIC}{read.RIC}}, \code{\LinkA{clean.RIC}{clean.RIC}}
\end{SeeAlso}
\begin{Examples}
\begin{ExampleCode}
data(RIC)
cleaned.data <- clean.RIC(RIC)
bouts <- bouts.RIC(cleaned.data)
#With 8 minutes bout difference
bouts <- bouts.RIC(cleaned.data,bout.diff=8)
\end{ExampleCode}
\end{Examples}

