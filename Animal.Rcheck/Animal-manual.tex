\documentclass{book}
\usepackage[times,hyper]{Rd}
\begin{document}
\HeaderA{bouts.RIC}{Merge bouts from RIC roughage intake files}{bouts.RIC}
\begin{Description}\relax
Merges single visits from roughage intake log file if the time
difference between successive visits is less than specified time
difference i.e merges multiple rows in the log file that are considered
to be a single feeding bout to a single row.
\end{Description}
\begin{Usage}
\begin{verbatim}
bouts.RIC(data, bout.diff = 5)
\end{verbatim}
\end{Usage}
\begin{Arguments}
\begin{ldescription}
\item[\code{data}] A data.frame read in with read.RIC, with clean=T option
\item[\code{bout.diff}] The maximum time difference (in minutes) between visits in a
single bout
\end{ldescription}
\end{Arguments}
\begin{Value}
A data.frame with the values merged for individual bouts. All other
objects are self.explanatory, but there are two that need further
clarification  
\begin{ldescription}
\item[\code{bout.duration}] This is the duration in minutes of the merged bout
i.e. begin-end for the bout.
\item[\code{intake.duration}] This is the time in minutes that the cow has kept her head
in the feeding trough during the bout. You may want to use this for
calculating feeding speed (kg/min)
\end{ldescription}
\end{Value}
\begin{Note}\relax
The function is currently only implemented for merging feeding
time and intake, thus protein, ash etc. are dropped from the resulting
datafrane. The variables in the code are finnish and thus maybe difficult to
follow. An english translation will (possibly) appear in the future.
\end{Note}
\begin{Author}\relax
Matti Pastell <matti.pastell@helsinki.fi>
\end{Author}
\begin{SeeAlso}\relax
\code{\LinkA{read.RIC}{read.RIC}}, \code{\LinkA{clean.RIC}{clean.RIC}}
\end{SeeAlso}
\begin{Examples}
\begin{ExampleCode}
data(RIC)
cleaned.data <- clean.RIC(RIC)
bouts <- bouts.RIC(cleaned.data)
#With 8 minutes bout difference
bouts <- bouts.RIC(cleaned.data,bout.diff=8)
\end{ExampleCode}
\end{Examples}

\HeaderA{clean.RIC}{Clean RIC roughage intake log file}{clean.RIC}
\begin{Description}\relax
Performs the following clean ups on RIC roughage intake files:
Removes lines with Cow number 0 and lines with negative feed intake
and visits with 0 duration. Equal using clean=T with read.RIC.
\end{Description}
\begin{Usage}
\begin{verbatim}
clean.RIC(data)
\end{verbatim}
\end{Usage}
\begin{Arguments}
\begin{ldescription}
\item[\code{data}] A data.frame read in with read.RIC
\end{ldescription}
\end{Arguments}
\begin{Value}
Cleaned data.frame
\end{Value}
\begin{Author}\relax
Matti Pastell <matti.pastell@helsinki.fi>
\end{Author}
\begin{SeeAlso}\relax
\code{\LinkA{read.RIC}{read.RIC}}, \code{\LinkA{bouts.RIC}{bouts.RIC}}
\end{SeeAlso}
\begin{Examples}
\begin{ExampleCode}
data(RIC)
cleaned.data <- clean.RIC(RIC)
\end{ExampleCode}
\end{Examples}

\HeaderA{cowAnalyze}{Analyze time coded behavior data}{cowAnalyze}
\begin{Description}\relax
This function provides descriptive statistics from time coded behavior
datafiles recorded with CowLog.
\end{Description}
\begin{Usage}
\begin{verbatim}
cowAnalyze(file = NULL, states = NULL, events = NULL,
 state.names = NULL, event.names = NULL)
\end{verbatim}
\end{Usage}
\begin{Arguments}
\begin{ldescription}
\item[\code{file}] CowLog data file, or a file in same format
\item[\code{states}] A vector with the codes in the file that belong to states
\item[\code{events}] A vector with codes in the file that belong to events
\item[\code{state.names}] A character vector with the names for the states 
\item[\code{event.names}] A character vector with the names for the events
\end{ldescription}
\end{Arguments}
\begin{Value}
\begin{ldescription}
\item[\code{state}] Results for states
\item[\code{event}] Results for events
\end{ldescription}
\end{Value}
\begin{Author}\relax
Matti Pastell <matti.pastell@helsinki.fi>
\end{Author}
\begin{References}\relax
Hanninen, L. \& Pastell, M. CowLog: Open source software for coding
behaviors from digital video. Behavior Research Methods. 41(2),
472-476.

http://www.mm.helsinki.fi/~mpastell/CowLog
\end{References}
\begin{Examples}
\begin{ExampleCode}
##Analyze CowLog datafile named calf1.bh1,
## codes 1-3 are states and codes 4-5 are states.
## The names for the states are lying, standing, walking.
## Not run: 
analyzed <-cowAnalyze(file='calf1.bh1',states=c(1,2,3),
events=c(4,5),state.names=c('lying','standing','walking'))
## End(Not run)
 
\end{ExampleCode}
\end{Examples}

\HeaderA{daily}{Calculate daily values from time series}{daily}
\begin{Description}\relax
Calculate daily values (e.g mean or sum) from time series data. Allows
to specify a subject to calculate daily values for several subjects.
\end{Description}
\begin{Usage}
\begin{verbatim}
daily(data, time, fun = sum, subject = NULL)
\end{verbatim}
\end{Usage}
\begin{Arguments}
\begin{ldescription}
\item[\code{data}] A data vector that you want to calculate the daily values
for
\item[\code{time}] Time stamps for data in POSIXct format
\item[\code{fun}] The function to apply, defaults to sum
\item[\code{subject}] You can optionally specify to a subject. e.g. to get
daily values for each cow in a herd.
\end{ldescription}
\end{Arguments}
\begin{Value}
A data.frame with following elements
\begin{ldescription}
\item[\code{Day}] Date
\item[\code{Subject}] Appears only if you have specified a subject
\item[\code{Result}] The result of the function
\end{ldescription}
\end{Value}
\begin{Author}\relax
Matti Pastell <matti.pastell@helsinki.fi>
\end{Author}
\begin{SeeAlso}\relax
\code{\LinkA{monthly}{monthly}}, \code{\LinkA{hourly}{hourly}},
\code{\LinkA{weekly}{weekly}}
\end{SeeAlso}
\begin{Examples}
\begin{ExampleCode}
data(RIC)
RIC2 <- clean.RIC(RIC)
#Daily feed intake of a whole from data set RIC
herd <- daily(RIC2$intake,time=RIC2$begin,fun=sum)
#Daily feed intake of individual cows from data set RIC
herd <- daily(RIC2$intake,time=RIC2$begin,fun=sum,subject=RIC2$cowID)
\end{ExampleCode}
\end{Examples}

\HeaderA{day.string}{Convert dates to string}{day.string}
\begin{Description}\relax
This function converts POSIXt dates to a string reprentation of the
date. The string format is convenient when calculating daily summaries etc.
\end{Description}
\begin{Usage}
\begin{verbatim}
day.string(x)
\end{verbatim}
\end{Usage}
\begin{Arguments}
\begin{ldescription}
\item[\code{x}] A POSIXt object
\end{ldescription}
\end{Arguments}
\begin{Value}
Day in string format.
\end{Value}
\begin{Author}\relax
Matti Pastell <matti.pastell@helsinki.fi>
\end{Author}
\begin{SeeAlso}\relax
\code{\LinkA{day}{day}}, \code{\LinkA{hour}{hour}},
\code{\LinkA{week}{week}}, \code{\LinkA{month}{month}}
\end{SeeAlso}
\begin{Examples}
\begin{ExampleCode}
date <- Sys.time()
day.str <- day.string(date)
print(day.str)
\end{ExampleCode}
\end{Examples}

\HeaderA{day}{Convert dates to day numbers}{day}
\begin{Description}\relax
This function extracts the day of month from date objects
\end{Description}
\begin{Arguments}
\begin{ldescription}
\item[\code{x}] A POSIXt object
\end{ldescription}
\end{Arguments}
\begin{Value}
\begin{ldescription}
\item[\code{day}] Day of month for the input object
\end{ldescription}
\end{Value}
\begin{Author}\relax
Matti Pastell <matti.pastell@helsinki.fi>
\end{Author}
\begin{SeeAlso}\relax
\code{\LinkA{week}{week}}, \code{\LinkA{day.string}{day.string}},
\code{\LinkA{hour}{hour}}, \code{\LinkA{month}{month}}
\end{SeeAlso}
\begin{Examples}
\begin{ExampleCode}
date <- Sys.time()
day.number <- day(date)
print(day.number)
\end{ExampleCode}
\end{Examples}

\HeaderA{delete.duplicates}{Delete duplicated state events}{delete.duplicates}
\begin{Description}\relax
Used internally by cowAnalyze
\end{Description}
\begin{Usage}
\begin{verbatim}
delete.duplicates(obs)
\end{verbatim}
\end{Usage}
\begin{Arguments}
\begin{ldescription}
\item[\code{obs}] A dataframe containing state events
\end{ldescription}
\end{Arguments}
\begin{Author}\relax
Matti Pastell
\end{Author}

\HeaderA{freq.count}{Count frequency of behaviors}{freq.count}
\begin{Description}\relax
Used internally by cowAnalyze
\end{Description}
\begin{Usage}
\begin{verbatim}
freq.count(x)
\end{verbatim}
\end{Usage}
\begin{Arguments}
\begin{ldescription}
\item[\code{x}] A numeric or factor vector of behaviors
\end{ldescription}
\end{Arguments}
\begin{Author}\relax
Matti Pastell
\end{Author}

\HeaderA{hour}{Convert times to hours}{hour}
\begin{Description}\relax
This function extracts the hour from date objects
\end{Description}
\begin{Arguments}
\begin{ldescription}
\item[\code{x}] A POSIXt object
\end{ldescription}
\end{Arguments}
\begin{Value}
\begin{ldescription}
\item[\code{hour}] Hour (1-24) of the input object
\end{ldescription}
\end{Value}
\begin{Author}\relax
Matti Pastell <matti.pastell@helsinki.fi>
\end{Author}
\begin{SeeAlso}\relax
\code{\LinkA{day}{day}}, \code{\LinkA{day.string}{day.string}},
\code{\LinkA{week}{week}}, \code{\LinkA{month}{month}}
\end{SeeAlso}
\begin{Examples}
\begin{ExampleCode}
date <- Sys.time()
hour.number <- hour(date)
print(hour.number)
\end{ExampleCode}
\end{Examples}

\HeaderA{hourly}{Calculate hourly values from time series}{hourly}
\begin{Description}\relax
Calculate hourly values (e.g mean or sum) from time series data. Allows
to specify a subject to calculate hourly values for several subjects.
\end{Description}
\begin{Usage}
\begin{verbatim}
hourly(data, time, fun = sum, subject = NULL)
\end{verbatim}
\end{Usage}
\begin{Arguments}
\begin{ldescription}
\item[\code{data}] A data vector that you want to calculate the hourly values
for
\item[\code{time}] Time stamps for data in POSIXct format
\item[\code{fun}] The function to apply, defaults to sum
\item[\code{subject}] You can optionally specify to a subject. e.g. to get
hourly values for each cow in a herd.
\end{ldescription}
\end{Arguments}
\begin{Value}
A data.frame with following elements
\begin{ldescription}
\item[\code{Hour}] Hour 1-24
\item[\code{Subject}] Appears only if you have specified a subject
\item[\code{Result}] The result of the function
\end{ldescription}
\end{Value}
\begin{Author}\relax
Matti Pastell <matti.pastell@helsinki.fi>
\end{Author}
\begin{SeeAlso}\relax
\code{\LinkA{daily}{daily}}, \code{\LinkA{weekly}{weekly}},
\code{\LinkA{monthly}{monthly}}
\end{SeeAlso}
\begin{Examples}
\begin{ExampleCode}
data(RIC)
RIC2 <- clean.RIC(RIC)
#Hourly feed intake of a whole from data set RIC
herd <- hourly(RIC2$intake,time=RIC2$begin,fun=sum)
#Hourly feed intake of individual cows from data set RIC
herd <- hourly(RIC2$intake,time=RIC2$begin,fun=sum,subject=RIC2$cowID)
\end{ExampleCode}
\end{Examples}

\HeaderA{label.data}{Label measurement data}{label.data}
\begin{Description}\relax
This function labels measurement data according to behavioral
observations. The format of behavioral observations follows CowLog
convention (See details).
\end{Description}
\begin{Usage}
\begin{verbatim}
label.data(data, observation, db = 5, de = 5, min.length = 20)
\end{verbatim}
\end{Usage}
\begin{Arguments}
\begin{ldescription}
\item[\code{data}] A data.frame holding the measurement data, one of the
elements needs to be named time and hold the time stamp for each row.
\item[\code{observation}] A data.frame with behavioral observations, one of
the elements needs to be named time and hold the time stamps for the
behaviors and another named behavior and hold the labels for
respective behaviors. See details.
\item[\code{db}] Specifies the delay in seconds from the observation to the start of
labeling, this is sometimes useful if the data is used for model
building and we want to eliminate the temporal inaccury due to
behavioral observations. Defaults to 5
\item[\code{de}] Similarly to db, specifies the (negative) delay in seconds
from the end of observation. Defaults to 5
\item[\code{min.length}] descibes the minimum length of
the labeled data vector. Again sometimes we want to have long enough
data vectors for model building and leave out too short bits. This is
the length after substracting db and de. Defaults to 20
\end{ldescription}
\end{Arguments}
\begin{Details}\relax
The time stamp in behavior and data need to be in the same format i.e. POSIXt or seconds from the start in numeric
format. The POSIXt is naturally more convinient since then the
behavioral observation and the measurement need not to begin from the
same time point.
The data.frame observation need to have at least two elements: time and
behavior (any additional elements are ignored). The time specifies the
start of the corresponding behavior and the start time of the next
behavior is used as the end of the previous one.
The format is adopted from CowLog behavioral coding software
(http://www.mm.helsinki.fi/~mpastell/CowLog).
\end{Details}
\begin{Value}
The original data with two additional elements: \code{label} which contains
the labels for each row in the data and \code{freq} which tells the number of
the label counting from the beginning.
\end{Value}
\begin{Author}\relax
Matti Pastell <matti.pastell@helsinki.fi>
\end{Author}

\HeaderA{month}{Convert dates to month numbers}{month}
\begin{Description}\relax
This function extracts the day of month from date objects
\end{Description}
\begin{Arguments}
\begin{ldescription}
\item[\code{x}] A POSIXt object
\end{ldescription}
\end{Arguments}
\begin{Value}
\begin{ldescription}
\item[\code{month}] Month number of the input object
\end{ldescription}
\end{Value}
\begin{Author}\relax
Matti Pastell <matti.pastell@helsinki.fi>
\end{Author}
\begin{SeeAlso}\relax
\code{\LinkA{day}{day}}, \code{\LinkA{day.string}{day.string}},
\code{\LinkA{hour}{hour}},\code{\LinkA{week}{week}}
\end{SeeAlso}
\begin{Examples}
\begin{ExampleCode}
date <- Sys.time()
month.number <- month(date)
print(month)
\end{ExampleCode}
\end{Examples}

\HeaderA{monthly}{Calculate monthly values from time series}{monthly}
\begin{Description}\relax
Calculate monthly values (e.g mean or sum) from time series data. Allows
to specify a subject to calculate monthly values for several subjects.
\end{Description}
\begin{Usage}
\begin{verbatim}
monthly(data, time, fun = sum, subject = NULL)
\end{verbatim}
\end{Usage}
\begin{Arguments}
\begin{ldescription}
\item[\code{data}] A data vector that you want to calculate the monthly values
for
\item[\code{time}] Time stamps for data in POSIXct format
\item[\code{fun}] The function to apply, defaults to sum
\item[\code{subject}] You can optionally specify to a subject. e.g. to get
monthly values for each cow in a herd.
\end{ldescription}
\end{Arguments}
\begin{Value}
A data.frame with following elements
\begin{ldescription}
\item[\code{Day}] Date
\item[\code{Subject}] Appears only if you have specified a subject
\item[\code{Result}] The result of the function
\end{ldescription}
\end{Value}
\begin{Author}\relax
Matti Pastell <matti.pastell@helsinki.fi>
\end{Author}
\begin{SeeAlso}\relax
\code{\LinkA{monthly}{monthly}}, \code{\LinkA{hourly}{hourly}},
\code{\LinkA{weekly}{weekly}}
\end{SeeAlso}
\begin{Examples}
\begin{ExampleCode}
data(RIC)
RIC2 <- clean.RIC(RIC)
#Monthly feed intake of a whole from data set RIC
herd <- monthly(RIC2$intake,time=RIC2$begin,fun=sum)
#Monthly feed intake of individual cows from data set RIC
herd <- monthly(RIC2$intake,time=RIC2$begin,fun=sum,subject=RIC2$cowID)
\end{ExampleCode}
\end{Examples}

\HeaderA{nunique}{Count unique occurrences of variables}{nunique}
\begin{Description}\relax
Returns the number of unique occurrences of each level in the input object.
\end{Description}
\begin{Usage}
\begin{verbatim}
nunique(x)
\end{verbatim}
\end{Usage}
\begin{Arguments}
\begin{ldescription}
\item[\code{x}] Numeric, character of factor vector
\end{ldescription}
\end{Arguments}
\begin{Details}\relax
Provides a convenient way to calculate the unique occurrences of certain
events in daily, hourly, weekly and monthly data e.g. calculate the
number of unique animals that have used the feeding throughs each hour
in dataset RIC (see examples).
\end{Details}
\begin{Value}
Number of unique levels in the input object.
\end{Value}
\begin{Author}\relax
Matti Pastell <matti.pastell@helsinki.fi>
\end{Author}
\begin{Examples}
\begin{ExampleCode}
#Lets count the number of unique cows that have started to eat each hour
#in the dataset RIC.
data(RIC)
data <- clean.RIC(RIC)
hourly(RIC$cowID,RIC$begin,nunique)

\end{ExampleCode}
\end{Examples}

\HeaderA{partOfDay}{Code data into different parts of day}{partOfDay}
\begin{Description}\relax
This function returns the part of day
from time stamps. The day can be split into parts of different length
with a chosen start time for the splits.
\end{Description}
\begin{Usage}
\begin{verbatim}
partOfDay(time, nsplit = 4, start = 1)
\end{verbatim}
\end{Usage}
\begin{Arguments}
\begin{ldescription}
\item[\code{time}] Timestamp vector in POSIXct format
\item[\code{nsplit}] Number of splits.
\item[\code{start}] Start time of the split in hours (1-24)
\end{ldescription}
\end{Arguments}
\begin{Details}\relax
It is often useful to observe the amount of behaviors during different
part of day e.g. if we want to find out how different behaviours are
distributed over the entire day. This function returns the part of day
from time stamps. The function returns only even hours, if nsplit
provides intervals with decimal hours they will be rounded to
nearest integer.
\end{Details}
\begin{Value}
A factor with the part of day for input timestamps with hour intervals
as labels-
\end{Value}
\begin{Author}\relax
Matti Pastell <matti.pastell@helsinki.fi>
\end{Author}
\begin{SeeAlso}\relax
\code{\LinkA{hour}{hour}}, \code{\LinkA{hourly}{hourly}}
\end{SeeAlso}
\begin{Examples}
\begin{ExampleCode}
#Look at the daily distribution of feed intake of cows
#in dataset RIC
data(RIC)
data <- clean.RIC(RIC)
#With default split
data$period <- partOfDay(data$begin)
#Plot the results
boxplot(intake~period,data=data,ylab='Feed intake (kg)',
xlab='Time of day',main='Default settings: start =1, nsplit=4')
#A different split with directly plotting the result
boxplot(intake~partOfDay(begin,nsplit=6,start=3),data=data,
ylab='Feed intake (kg)',xlab='Time of day',main='start=3,nsplit=6')
\end{ExampleCode}
\end{Examples}

\HeaderA{read.RIC}{Read RIC feed measurement system log files}{read.RIC}
\begin{Description}\relax
Reads in roughage intake log files produced by the Insentec
RIC-Management Windows software. (VRyymmdd.DAT) The function converts
the start and end times to POSIXct  objects and adds the date to each
time stamp from the file name.
\end{Description}
\begin{Usage}
\begin{verbatim}
read.RIC(file, clean = TRUE)
\end{verbatim}
\end{Usage}
\begin{Arguments}
\begin{ldescription}
\item[\code{file}] The roughage intake log file, (VRyymmdd.DAT)
\item[\code{clean}] If true the function removes lines with Cow number 0 and
lines with negative feed intake and visits with 0 duration.
Values TRUE of FALSE, Defaults to TRUE
\end{ldescription}
\end{Arguments}
\begin{Value}
A data.frame with the formatted insentec data.
\end{Value}
\begin{Author}\relax
Matti Pastell <matti.pastell@helsinki.fi>
\end{Author}
\begin{References}\relax
B.V. Marknesse. Instructions for use. RIC - MANAGEMENT WINDOWS version
RW:1.7. English. Insentec
\end{References}
\begin{SeeAlso}\relax
\code{\LinkA{clean.RIC}{clean.RIC}}, \code{\LinkA{bouts.RIC}{bouts.RIC}}
\end{SeeAlso}
\begin{Examples}
\begin{ExampleCode}
## Not run: data <- read.RIC('VR080811.DAT')
\end{ExampleCode}
\end{Examples}

\HeaderA{RIC}{RIC roughage intake log}{RIC}
\keyword{datasets}{RIC}
\begin{Description}\relax
A roughage intake log of a one day from Insentec RIC feed measurement
system from the University of Helsinki Viikki researc barn in Finland.
The data has been imported to R with function read.RIC option clean=F
\end{Description}
\begin{Usage}
\begin{verbatim}data(RIC)\end{verbatim}
\end{Usage}
\begin{Format}\relax
A data frame with 3242 observations on the following 16 variables.
\describe{
\item[\code{transponder}] Transponder number
\item[\code{cowID}] Cow number
\item[\code{trough}] The number of feed trough
\item[\code{begin}] Start time of the visit
\item[\code{end}] End time of the visit
\item[\code{duration}] Visit duration in seconds
\item[\code{begin.kg}] Roughage before visit
\item[\code{end.kg}] Roughage after visit
\item[\code{feed.type}] Type of feed
\item[\code{intake}] Feed intake (kg)
\item[\code{DM}] Dry matter (kg)
\item[\code{energy}] Energy (VEM)
\item[\code{protein}] Protein (kg)
\item[\code{crude.fibre}] Crude fibre (kg)
\item[\code{fat}] Fat (kg)
\item[\code{ash}] Ash (kg)
}
\end{Format}
\begin{References}\relax
B.V. Marknesse. Instructions for use. RIC - MANAGEMENT WINDOWS version
RW:1.7. English. Insentec
\end{References}

\HeaderA{state.durations}{Calculate state durations}{state.durations}
\begin{Description}\relax
Used internally by cowAnalyze
\end{Description}
\begin{Usage}
\begin{verbatim}
state.durations(obs, state.names = NULL)
\end{verbatim}
\end{Usage}
\begin{Arguments}
\begin{ldescription}
\item[\code{obs}] A dataframe containing state events
\item[\code{state.names}] A character vector of state behavior names
\end{ldescription}
\end{Arguments}
\begin{Author}\relax
Matti Pastell
\end{Author}

\HeaderA{week}{Convert dates to week numbers}{week}
\begin{Description}\relax
This function extracts the week number from date objects
\end{Description}
\begin{Arguments}
\begin{ldescription}
\item[\code{x}] A POSIXt object
\end{ldescription}
\end{Arguments}
\begin{Value}
\begin{ldescription}
\item[\code{week}] Week number of the input object
\end{ldescription}
\end{Value}
\begin{Author}\relax
Matti Pastell <matti.pastell@helsinki.fi>
\end{Author}
\begin{SeeAlso}\relax
\code{\LinkA{day}{day}}, \code{\LinkA{day.string}{day.string}},
\code{\LinkA{hour}{hour}}, \code{\LinkA{month}{month}}
\end{SeeAlso}
\begin{Examples}
\begin{ExampleCode}
date <- Sys.time()
week.number <- week(date)
print(week.number)
\end{ExampleCode}
\end{Examples}

\HeaderA{weekly}{Calculate weekly values from time series}{weekly}
\begin{Description}\relax
Calculate weekly values (e.g mean or sum) from time series data. Allows
to specify a subject to calculate weekly values for several subjects.
\end{Description}
\begin{Usage}
\begin{verbatim}
weekly(data, time, fun = sum, subject = NULL)
\end{verbatim}
\end{Usage}
\begin{Arguments}
\begin{ldescription}
\item[\code{data}] A data vector that you want to calculate the weekly values
for
\item[\code{time}] Time stamps for data in POSIXct format
\item[\code{fun}] The function to apply, defaults to sum
\item[\code{subject}] You can optionally specify to a subject. e.g. to get
weekly values for each cow in a herd.
\end{ldescription}
\end{Arguments}
\begin{Value}
A data.frame with following elements
\begin{ldescription}
\item[\code{Week}] Week number
\item[\code{Subject}] Appears only if you have specified a subject
\item[\code{Result}] The result of the function
\end{ldescription}
\end{Value}
\begin{Author}\relax
Matti Pastell <matti.pastell@helsinki.fi>
\end{Author}
\begin{SeeAlso}\relax
\code{\LinkA{daily}{daily}}, \code{\LinkA{hourly}{hourly}},
\code{\LinkA{monthly}{monthly}}
\end{SeeAlso}
\begin{Examples}
\begin{ExampleCode}
data(RIC)
RIC2 <- clean.RIC(RIC)
#Weekly feed intake of a whole from data set RIC
herd <- weekly(RIC2$intake,time=RIC2$begin,fun=sum)
#Weekly feed intake of individual cows from data set RIC
herd <- weekly(RIC2$intake,time=RIC2$begin,fun=sum,subject=RIC2$cowID)
\end{ExampleCode}
\end{Examples}

\end{document}
